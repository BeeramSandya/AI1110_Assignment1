\documentclass[twocolumn,24pt]{IEEEtran}
\usepackage[utf8]{inputenc}
\usepackage{amsmath}
\begin{document}
\title{Assignment1}
\author{\Large Beeram Sandya\\CS21BTECH11006}
\maketitle

\section{\Large ICSE 10 2018 Paper\\Question 1(c)}
\Large Cards bearing numbers 2, 4, 6, 8, 10, 12, 14, 16, 18 and 20 are kept in a bag. A card is drawn at random from the bag. Find the probability of getting a card which is:\\ 
(i) a prime number.\\
(ii) a number divisible by 4.\\
(iii) a number that is a multiple of 6.\\
(iv) an odd number.\\
\section{ \LARGE Answer}
 Probability, by definition is the ratio of number of favourable outcomes to total number of possible outcomes.\\\\\\
(i) As given numbers are all even numbers from 2 to 20, only 2 is the prime number. And number of possible outcomes is 10.\\\\
So required probability of getting a card bearing a prime number is \( \frac{1}{10} \).\\\\\\\\
(ii) Among the given numbers 4, 8, 12, 16 and 20 are divisible by 4. The total number of cards are 10 out of which 5 cards are divisible by 4.\\\\
Required probability is \( \frac{5}{10} \) = \( \frac{1}{2} \).
\\\\ Therefore, probability of getting a card bearing a number that is divisible by 4 is \(\frac{1}{2} \).\\\\\\\\
(iii) Among the given numbers 6, 12 and 18 are multiples of 6.The total number of outcomes are 10 out of which 3 are required cards.\\\\
So, required probability of getting a card bearing a number that is multiple of 6 is \(\frac{3}{10}\).\\\\\\\\
(iv) There are no odd numbers present among the given set of numbers.\\\\
Therefore, the probability of getting a card bearing an odd number is  0.

\end{document}
